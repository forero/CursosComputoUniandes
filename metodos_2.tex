\documentclass[letterpaper,10pt,onecolumn]{article}
\usepackage[spanish]{babel}
\usepackage[utf8]{inputenc}
\usepackage{amsfonts}
\usepackage{amsthm}
\usepackage{amsmath}
\usepackage{mathrsfs}

\usepackage{enumitem}
\usepackage[pdftex]{color,graphicx}
\usepackage{hyperref}
\usepackage{listings}
\usepackage{calligra}
\usepackage{url}
%\usepackage{algpseudocode} 
\DeclareMathAlphabet{\mathcalligra}{T1}{calligra}{m}{n}
\DeclareFontShape{T1}{calligra}{m}{n}{<->s*[2.2]callig15}{}
\newcommand{\scripty}[1]{\ensuremath{\mathcalligra{#1}}}
\lstloadlanguages{[5.2]Mathematica}
\setlength{\oddsidemargin}{0cm}
\setlength{\textwidth}{490pt}
\setlength{\topmargin}{-40pt}
\addtolength{\hoffset}{-0.3cm}
\addtolength{\textheight}{4cm}

\begin{document}
\begin{center}

\includegraphics[width=490pt]{header.png}\\[0.5cm]

\textsc{\LARGE M\'etodos Computacionales 2}\\[0.1cm]
%\large Nombre Profesora \\[0.5cm]

\end{center}

\large \noindent\textsc{Nombre del curso:}  M\'etodos Computacionales 2 % aqui %nombre del curso 
  
\noindent\textsc{C\'odigo del curso:}  %Aqui el codigo del %curso 

\noindent\textsc{Unidad acad\'emica:} Departamento de F\'isica

\noindent\textsc{Periodo acad\'emico:} %Aqui el periodo, %p.ej. 201510 

\noindent\textsc{Horario:}
%Aqui el horario, %p.ej. Ma y Ju, 10:00 a 11:20 

\noindent\rule{\textwidth}{1pt}\\[-0.3cm]

\normalsize \noindent\textsc{Nombre profesor magistral:}
%Aqui nombre del profesor principal   

\noindent\textsc{Correo electr\'onico:}
\href{mailto:nn@uniandes.edu.co}{\nolinkurl{nn@uniandes.edu.co}}
%Cambie address por su direccion de correo uniandes 

\noindent\textsc{Horario y lugar de atenci\'on:} 
% horario de atencion
\\[-0.1cm]




\normalsize \noindent\textsc{Nombre profesor Complementaria:}
%Aqui nombre del profesor principal 

\noindent\textsc{Correo electr\'onico:}
\href{mailto:nn@uniandes.edu.co}{\nolinkurl{nn@uniandes.edu.co}}
%Cambie address por su direccion de correo uniandes 

\noindent\textsc{Horario de atenci\'on:} con cita previa. 
\\[-0.1cm]
%\href{mailto: jd.prada1760@uniandes.edu.co}{\nolinkurl{jd.prada1760@uniandes.edu.co}}

%Cambie address por direccion de correo uniandes del profesor
%complementario 

%\noindent\textsc{Horario y lugar de atenci\'on:} %Aqui horario y
%lugar de atencion del profesor complementario, p.ej. Vi, 15:00 a
%17:00, Oficina Ip102 
%\\[-0.1cm]
%Repetir esto en caso de varios profesores complementarios

%\noindent\textsc{Nombre monitor(a):} %Aqui nombre del monitor si aplica

%\noindent\textsc{Correo electr\'onico:}
%\href{mailto:address@uniandes.edu.co}{\nolinkurl{address@uniandes.edu.co}}
%%Cambie address por direccion de correo uniandes del monitor 

%\noindent\textsc{Horario y lugar de atenci\'on:} %Aqui horario y
%lugar de atencion del monitor, p.ej. Vi, 15:00 a 17:00, Oficina Ip102 

\noindent\rule{\textwidth}{1pt}\\[-0.1cm]

\newcounter{mysection}
\addtocounter{mysection}{1}

\noindent\textbf{\large \Roman{mysection} \quad Introducci\'on}\\[-0.2cm]

%Este espacio es para hacer una introduccion al curso, evidenciando la
%propuesta metodologica. Debe ser clara y precisa. 

\noindent\normalsize Los m\'etodos computacionales son un aspecto
inseparable de cualquier \'area de trabajo en ciencia e ingenier\'ia.
Esto se debe a la facilidad de acceso a 
computadoras programables  y al aumento exponencial en su capacidad de
procesamiento con la aparici\'on de los procesadores multi--n\'ucleo y las FPGA's.
Esta diversificaci\'on de recursos para la realizaci\'on de c\'alculos presenta muchos desaf\'{\i}os para la investigaci\'on
y, por ende, vemos la necesidad de democratizar el conocimiento para formar
cient\'{\i}ficos vers\'atiles capaces de utilizar la tecnolog\'{\i}a
de manera eficiente y as\'{\i} poder dar respuesta a las necesidades de la sociedad. 
Este primer curso de la serie computacional de F\'{\i}sica busca 
\\[0.1cm] 

\stepcounter{mysection}
\noindent\textbf{\large \Roman{mysection} \quad Objetivos}\\[-0.2cm]

%En este espacio se debe precisar el ente visor del curso y el
%proposito ideal al finalizar el curso. 
\noindent\normalsize El objetivo principal del curso es presentar
algoritmos y t\'ecnicas b\'asicas para:

\begin{itemize}
\item analizar y describir datos con t\'ecnicas estad\'isticas basadas
  en m\'etodos Monte Carlo, \\[-0.6cm]
\item desarrollar esquemas reproducibles para el an\'alisis de datos cient\'ificos. \\[-0.6cm]
\end{itemize} 
\vspace*{0.5cm} 

\stepcounter{mysection}
\noindent\textbf{\large \Roman{mysection} \quad Competencias a
  desarrollar}\\[-0.2cm] 

%En este espacio se describen las habilidades que el estudiante desarrollara en el transcurso del curso.

\noindent\normalsize Al finalizar el curso, se espera que el
estudiante est\'e en capacidad de: 

\begin{itemize}
\item implementar , \\[-0.6cm]   
\item manejar ,\\[-0.6cm]  
\item desarrollar .\\[-0.6cm]  
\end{itemize}

\vspace*{0.5cm} 

\stepcounter{mysection}
\noindent\textbf{\large \Roman{mysection} \quad Contenido por
  semanas}\\[-0.2cm]  

%Se expone de forma ordenada toda la tematica a tratar del curso. Debe planearse para 15 semanas.

\noindent\textbf{\textsc{Semana 1}}\\[-0.5cm]
\begin{itemize}
\item Temas: C++. Introducción, sintaxis, compilar/ejecutar,
variables, ciclos, If/while. \\[-0.6cm]
\item Lecturas preparatorias: C++ Tutorial\\[-0.6cm]
\end{itemize}


\noindent\textbf{\textsc{Semana 2}}\\[-0.5cm]
\begin{itemize}
\item Temas: C++. Funciones, arreglos, punteros, clases. Makefiles \\[-0.6cm]
\item Lecturas preparatorias: C++ Tutorial. Videos de Sofware
  Carpentry sobre Makefiles.\\[-0.6cm]  
\end{itemize}

\noindent\textbf{\textsc{Semana 3}}\\[-0.5cm]
\begin{itemize}
\item Temas: Transformadas de Fourier. Se\~nales y Filtros. \\[-0.6cm]
\item Lecturas preparatorias: Cap\'itulo 10 (Fourier Analysis) del
  libro de Landau.\\[-0.6cm] 
\end{itemize}


\noindent\textbf{\textsc{Semana 4}}\\[-0.5cm]
\begin{itemize}
\item Temas: Ecuaciones diferenciales ordinarias de primer y segundo orden. \\[-0.6cm]
\item Lecturas preparatorias: Cap\'itulo 9 (ODEs) del libro de
  Landau. Cap\'itulo 2 del libro de Hutchinson. \\[-0.6cm] 
\end{itemize}

\noindent\textbf{\textsc{Semana 5}}\\[-0.5cm]
\begin{itemize}
\item Temas: Ecuaciones diferenciales parciales de primer orden. Ecuación de advección. \\[-0.6cm]
\item Lecturas preparatorias: Cap\'itulo 17 (PDEs) del libro de
  Landau. Cap\'tulo 4 del libro de Hutchinson. \\[-0.6cm] 
\end{itemize}


\noindent\textbf{\textsc{Semana 6}}\\[-0.5cm]
\begin{itemize}
\item Temas: Ecuaciones diferenciales parciales se segundo orden. Ecuaciones parabólicas.   Ecuaci\'on de Difusi\'on. \\[-0.6cm]
\item Lecturas preparatorias: Cap\'itulo 17 (PDEs) del libro de
  Landau. Cap\'itulo 5 de Hutchinson. \\[-0.6cm] 
\end{itemize}

\noindent\textbf{\textsc{Semana 7}}\\[-0.5cm]
\begin{itemize}
\item Temas: Ecuaciones diferenciales parciales de segundo orden. Ecuaciones elípticas. Ecuación de Poisson. \\[-0.6cm]
\item Lecturas preparatorias: Cap\'itulo 18 (PDEs) del libro de
  Landau. Cap\'itulo 6 de Hutchinson. \\[-0.6cm] 
\\[-0.6cm]
\end{itemize}

\noindent\textbf{\textsc{Semana 8}}\\[-0.5cm]
\begin{itemize}
\item Temas: Ecuaciones diferenciales parciales de segundo orden. Ecuaciones hiperbólicas. Din\'amica de Fluidos. \\[-0.6cm]
\item Lecturas preparatorias: Cap\'itulo 18 (PDEs) del libro de
  Landau. Cap\'itulo 7 de Hutchinson. \\[-0.6cm] 
\end{itemize}


\vspace*{0.5cm} 
\stepcounter{mysection}
\noindent\textbf{\large \Roman{mysection} \quad
  Metodolog\'ia}\\[-0.2cm] 

%Se describen las tecnicas y metodos para el desarrollo exitoso del curso.

\noindent\normalsize 
Magistral y Complementaria.


\vspace*{0.5cm} 
\stepcounter{mysection}
\noindent\textbf{\large \Roman{mysection} \quad Criterios de
  evaluaci\'on}\\[-0.2cm] 

Todos los ex\'amenes, talleres y ejercicios son
\textbf{individuales}.  
Si en las entregas se detecta que el trabajo no fue
individual (esto incluye colaboraci\'on con personas no inscritas en
el curso, i.e. a trav\'es de ``monitor\'ias'') se llevar\'a el caso a
comit\'e disciplinario y la nota del curso queda como Pendiente
Disciplinario hasta que el comit\'e tome alguna decisi\'on. 

Todas las entregas de talleres y ejercicios se har\'an a trav\'es de
SICUA.  {\bf No se aceptar\'a ninguna tarea por fuera de esa
  plataforma}, a menos que ocurra un una falla en los servidores de
SICUA que afecte a {\bf todos} los estudiantes del curso.


\vspace*{0.5cm} 

\stepcounter{mysection}
\noindent\textbf{\large \Roman{mysection} \quad
  Bibliograf\'ia}\\[-0.2cm] 

%Indicar los libros y la documentacion guia.


\noindent\normalsize Bibliograf\'ia principal:

\begin{itemize}

\item 
\textit{A student's guide to numerical methods}, Ian H. Hutchinson, Cambridge University Press, 2015.

\item
\textit{Software Carpentry: Python Testing}
\url{http://katyhuff.github.io/python-testing/}

\item
\textit{A survey of Computational Physics - Enlarged Python Book}
. R. H. Landau, M. J. P\'aez, C. C. Bordeianu. WILEY. 2012.
\url{https://psrc.aapt.org/items/detail.cfm?ID=11578}

\item
\textit{C++ programming for the absolute beginner.}
 M. Lee \& D. Henkemans, Second Edition, Cengage Learning, 2009.

\item
\textit{The C programming language.}
 B. Kernighan \& D. Ritchie, Second Edition, Prentice Hall.

\item Software Carpentry: \url{http://software-carpentry.org/}
\item C++ Tutorial: \url{https://www.tutorialspoint.com/cplusplus/}
\end{itemize}


\end{document}
