\documentclass[letterpaper,10pt,onecolumn]{article}
\usepackage[spanish]{babel}
\usepackage[utf8]{inputenc}
\usepackage{amsfonts}
\usepackage{amsthm}
\usepackage{amsmath}
\usepackage{mathrsfs}

\usepackage{enumitem}
\usepackage[pdftex]{color,graphicx}
\usepackage{hyperref}
\usepackage{listings}
\usepackage{calligra}
\usepackage{url}
%\usepackage{algpseudocode} 
\DeclareMathAlphabet{\mathcalligra}{T1}{calligra}{m}{n}
\DeclareFontShape{T1}{calligra}{m}{n}{<->s*[2.2]callig15}{}
\newcommand{\scripty}[1]{\ensuremath{\mathcalligra{#1}}}
\lstloadlanguages{[5.2]Mathematica}
\setlength{\oddsidemargin}{0cm}
\setlength{\textwidth}{490pt}
\setlength{\topmargin}{-40pt}
\addtolength{\hoffset}{-0.3cm}
\addtolength{\textheight}{4cm}

\begin{document}
\begin{center}

\includegraphics[width=490pt]{header.png}\\[0.5cm]

\textsc{\LARGE M\'etodos Computacionales 1}\\[0.1cm]
%\large Nombre Profesora \\[0.5cm]

\end{center}

\large \noindent\textsc{Nombre del curso:}  M\'etodos Computacionales 1 % aqui %nombre del curso 
  
\noindent\textsc{C\'odigo del curso:}  FISI-2026 %Aqui el codigo del %curso 

\noindent\textsc{Unidad acad\'emica:} Departamento de F\'isica

\noindent\textsc{Prerrequisitos}: Introducci\'on a la Programaci\'on (ISIS-1221). \'Algebra Lineal I (MATE-1105).

\noindent\textsc{Cr\'editos}: 3

\noindent\textsc{Periodo acad\'emico:} %Aqui el periodo, %p.ej. 201510 

\noindent\textsc{Horario:}
%Aqui el horario, %p.ej. Ma y Ju, 10:00 a 11:20 

\noindent\rule{\textwidth}{1pt}\\[-0.3cm]

\normalsize \noindent\textsc{Nombre profesor magistral:}
%Aqui nombre del profesor principal   

\noindent\textsc{Correo electr\'onico:}
\href{mailto:nn@uniandes.edu.co}{\nolinkurl{nn@uniandes.edu.co}}
%Cambie address por su direccion de correo uniandes 

\noindent\textsc{Horario y lugar de atenci\'on:} 
% horario de atencion
\\[-0.1cm]




\normalsize \noindent\textsc{Nombre profesor Complementaria:}
%Aqui nombre del profesor principal 

\noindent\textsc{Correo electr\'onico:}
\href{mailto:nn@uniandes.edu.co}{\nolinkurl{nn@uniandes.edu.co}}
%Cambie address por su direccion de correo uniandes 

\noindent\textsc{Horario de atenci\'on:} con cita previa. 
\\[-0.1cm]
%\href{mailto: jd.prada1760@uniandes.edu.co}{\nolinkurl{jd.prada1760@uniandes.edu.co}}

%Cambie address por direccion de correo uniandes del profesor
%complementario 

%\noindent\textsc{Horario y lugar de atenci\'on:} %Aqui horario y
%lugar de atencion del profesor complementario, p.ej. Vi, 15:00 a
%17:00, Oficina Ip102 
%\\[-0.1cm]
%Repetir esto en caso de varios profesores complementarios

%\noindent\textsc{Nombre monitor(a):} %Aqui nombre del monitor si aplica

%\noindent\textsc{Correo electr\'onico:}
%\href{mailto:address@uniandes.edu.co}{\nolinkurl{address@uniandes.edu.co}}
%%Cambie address por direccion de correo uniandes del monitor 

%\noindent\textsc{Horario y lugar de atenci\'on:} %Aqui horario y
%lugar de atencion del monitor, p.ej. Vi, 15:00 a 17:00, Oficina Ip102 

\noindent\rule{\textwidth}{1pt}\\[-0.1cm]

\newcounter{mysection}
\addtocounter{mysection}{1}

\noindent\textbf{\large \Roman{mysection} \quad Introducci\'on}\\[-0.2cm]

%Este espacio es para hacer una introduccion al curso, evidenciando la
%propuesta metodologica. Debe ser clara y precisa. 

\noindent\normalsize Los m\'etodos computacionales son un aspecto
inseparable de cualquier \'area de trabajo en ciencia e ingenier\'ia.
Esto se debe a la facilidad de acceso a 
computadoras programables  y al aumento exponencial en su capacidad de
procesamiento.
Esta diversificaci\'on de recursos para el c\'omputo s\'olo se puede aprovechar si 
los  cient\'{\i}ficos son capaces de utilizar la tecnolog\'{\i}a
de manera eficiente.
Adem\'as, la obtenci\'on de resultados cient\'ificos 
reproducibles requiere una comprensi\'on b\'asica de probabilidad y estad\'istica.
\\[0.1cm] 

\stepcounter{mysection}
\noindent\textbf{\large \Roman{mysection} \quad Objetivos}\\[-0.2cm]

%En este espacio se debe precisar el ente visor del curso y el
%proposito ideal al finalizar el curso. 
\noindent\normalsize El objetivo principal del curso es presentar
algoritmos y t\'ecnicas computacionales b\'asicas para:

\begin{itemize}
\item analizar y describir datos con m\'etodos estad\'isticos basadas
  en m\'etodos Monte Carlo. \\[-0.6cm]
\item resolver num\'ericamente problemas que involucren derivadas, integrales, sistemas de ecuaciones algebraicas, \\[-0.6cm]
\item desarrollar esquemas reproducibles para el an\'alisis de datos cient\'ificos. \\[-0.6cm]
\end{itemize} 
\vspace*{0.5cm} 

\stepcounter{mysection}
\noindent\textbf{\large \Roman{mysection} \quad Competencias a
  desarrollar}\\[-0.2cm] 

%En este espacio se describen las habilidades que el estudiante desarrollara en el transcurso del curso.

\noindent\normalsize Al finalizar el curso, se espera que el
estudiante adquiera las siguientes habilidades.

\begin{itemize}
\item Realizar an\'aisis de datos usando apropiadamente conceptos b\'asicos de probabilidad y estad\'istica en la soluci\'on de problemas en ciencias y otras \'areas: variables aleatorias, valores esperados, funciones binomiales y multinomiales, m\'etodos de Monte-Carlo, pruebas de hip\'otesis:
 \begin{itemize}
  \item Diferenciar distribuciones de probabilidad comunmente encontradas en ciencias: normal, poisson, log-normal, etc.
  \item Hacer histogramas para identificar la distribuci\'on de los datos como funci\'on de una variable espec\'fica.
  \item Estimar la moda, media, mediana, desviaci\'on estandar, curtosis y otros momentos de series de datos.
  \item Realizar pruebas cuantitativas para ver si dos series de datos corresponden a la misma distribuci\'on de probabilidad.
  \item Utilizar datos para proponer modelos param\'etricos y acotar los terminos libres de este. 
  \item Solucionar sistemas de ecuaciones algebraicas lineales. 
  \item Resolver problemas de m\'inicos cuadrados, planteados desde el sistema de ecuaciones lineales correspondiente.
  \item Realizar reducci\'on de dimensionalidad de datos a trav\'es del m\'etodo de componentes principales.
 \end{itemize}
\\[-0.6cm]   
\item Usar herranientas de Python para an\'alisis estad\'istico de datos, para la soluci\'on de problemas en ciencias y otras \'areas.\\[-0.6cm]
\begin{itemize}
   \item Implementar computacionalment modelos f\'isico--mat\'ematicos para el an\'alisis de datos utilizando \texttt{clases} y \texttt{funciones}.\\[-0.5cm]
   \item  Adoptar buenas pr\'acticas de escritura de c\'odigo para el desarrollo cont\'inuo de aplicaciones en ambientes colaborativos (comentarios, control de versiones, documentaci\'on).\\[-0.5cm]
    \item Utilizar de la biblioteca \texttt{pandas} para el manejo eficiente de los datos
    teniendo en cuenta los siguientes procedimientos: preprocesamiento, re--formato de
    columnas y normalizaci\'on.\\[-0.5cm]
   \item Utilizar la biblioteca \texttt{numpy} en la implementaci\'on de los algoritmos matem\'aticos que representen los diferentes sistemas en estudio.\\[-0.5cm]
    \item  Utilizar la biblioteca \texttt{matplotlib} para visualizar datos.\\[-0.5cm]
\end{itemize}\\[-0.7cm]
\item Solucionar numéricamente problemas sencillos de cálculo diferencial, cálculo integral y álgebra lineal: cálculo de integrales, cálculo de derivadas sistemas de ecuaciones lineales, integraci\'on de funciones, problemas de valores propios. \\[-0.6cm]  

\begin{itemize}
    \item Calcula num\'ericamente la primera y segunda derivada de funciones de una variable.
    \item Calcula num\'ericamente integrales unidimensionales definidas e indefinidas con los m\'etodos siguientes: trapecio, newton-rhapson y cuadratura gaussiana.
    \item Encuentra num\'ericamente la soluci\'on a sistemas de ecuaciones algebraicos lineales.
    \item Utiliza bibliotecas de \'algebra lineal para calcular valores y vectores propios.
    \item Interpreta los vectores y valores propios de una matriz de covarianza como herramientas para la reducci\'on de dimensionalidad de un conjunto de datos.
\end{itemize}
\end{itemize}
%de manera aut\'onoma la soluci\'on num\'erica a una variedad de fen\'omenos representados mediante ecuaciones diferenciales o distribuciones probabil\'{\i}sticas.

\vspace*{0.5cm} 

\newpage
\stepcounter{mysection}
\noindent\textbf{\large \Roman{mysection} \quad Contenido por
  semanas}\\[-0.2cm]  

%Se expone de forma ordenada toda la tematica a tratar del curso. Debe planearse para 15 semanas.

\noindent\textbf{\textsc{Semana 1}}\\[-0.5cm]
\begin{itemize}
\item Temas: 
Presentaci\'on del curso. Uso de Python en la nube. Comandos b\'asicos Unix. 
Repaso de la sintaxis b\'asica de Python. \\[-0.6cm] 
%\item Lecturas preparatorias: 
\end{itemize}


\noindent\textbf{\textsc{Semana 2}} \\[-0.5cm]
\begin{itemize}
\item Temas: 
Repaso de lectura de archivos, graficas, funciones y objetos en Python \\[-0.6cm] 
%\item Lecturas preparatorias: 
\end{itemize}

\noindent\textbf{\textsc{Semana 3}}\\[-0.5cm]
\begin{itemize}
\item Temas:  Introducci\'on a la probabilidad I: Conceptos fundamentales, interpretaci\'on de probabilidad, funciones de densidad. \\[-0.6cm]
%\item Lecturas preparatorias: \\[-0.6cm] 
\end{itemize}

\noindent\textbf{\textsc{Semana 4}}\\[-0.5cm]
\begin{itemize}
\item Temas:  Introducci\'on a la probabilidad II: funciones de variables aleatorias, valores esperados, propagaci\'on de errores, transformaci\'on ortogonal de variables aleatorias. \\[-0.6cm]
%\item Lecturas preparatorias: \\[-0.6cm]    
\end{itemize}

\noindent\textbf{\textsc{Semana 5}}\\[-0.5cm]
\begin{itemize}
\item Temas: Funciones de probabilidad: Funciones binomiales y multinomiales, distribuci\'on de Poisson, distribuci\'on uniforme. Distribuciones exponenciales, gaussianas, normales-logaritmicas, chi-cuadrado. \\[-0.6cm]
%\item Lecturas preparatorias: \\[-0.6cm]
\end{itemize}


\noindent\textbf{\textsc{Semana 6}}\\[-0.5cm]
\begin{itemize}
\item Temas:  M\'etodos de montecarlo: N\'umeros distribu\'idos aleatoriamente, el m\'etodo de transformaci\'on, m\'etodo de selecci\'on-rechazo. Metr\'opolis-Hastings. \\[-0.6cm]
%\item Lecturas preparatorias: \\[-0.6cm]  
\end{itemize}


\noindent\textbf{\textsc{Semana 7}}\\[-0.5cm]
\begin{itemize}
\item Temas: Pruebas estad\'isticas. Hip\'otesis, niveles de significancia. Ejemplo de construcci\'on de una prueba estad\'istica completa. Verificaci\'on de pruebas de ajuste. \\[-0.6cm]
%\item Lecturas preparatorias:  \\[-0.6cm] 
\end{itemize}

\noindent\textbf{\textsc{Semana 8}}\\[-0.5cm]
\begin{itemize}
\item Temas: Conceptos b\'asicos sobre estimaci\'on de par\'ametros. Muestras, estimadores y sesgo estad\'istico. Estimadores para la media, varianza y covarianza de datos. \\[-0.6cm]
%\item Lecturas preparatorias: \\[-0.6cm] 
\end{itemize}


\noindent\textbf{\textsc{Semana 9}}\\[-0.5cm]
\begin{itemize}
\item Temas: An\'alisis con estimadores de m\'axima verosimilitud. Introducci\'on a estimadores de MV y ejemplos. Varianza de estimadores MV. MV Extendida, MV con datos segmentados. Combinaci\'on de mediciones con MV. \\[-0.6cm]
%\item Lecturas preparatorias:
\\[-0.6cm]
\end{itemize}

\noindent\textbf{\textsc{Semana 10}}\\[-0.5cm]
\begin{itemize}
\item Temas: Conceptos b\'asicos de estad\'istica Bayesiana. Estimaci\'on de par\'ametros con estad\'istica bayesiana. \\[-0.6cm]  
%\item Lecturas preparatorias:\\[-0.6cm]
\end{itemize}


\noindent\textbf{\textsc{Semana 11}}\\[-0.5cm]
\begin{itemize}
\item Temas: Integraci\'on. \\[-0.6cm]
%\item Lecturas preparatorias: Cap\'itulo 6 (Integration) del libro de
%Landau.\\[-0.6cm]
\end{itemize}

\noindent\textbf{\textsc{Semana 12}}\\[-0.5cm]
\begin{itemize}
\item Temas: Derivadas. Ra\'ices de ecuaciones. \\[-0.6cm]
%\item Lecturas preparatorias: Cap\'itulos 7.I (Numerical
%  Differentiation) y 7.II (Trial-and-Error Searching) del libro de
%  Landau.\\[-0.6cm] 
\end{itemize}

\noindent\textbf{\textsc{Semana 13}}\\[-0.5cm]
\begin{itemize}
\item Temas: Interpolación. \\[-0.6cm]
%\item Lecturas: \\[-0.6cm] 
\end{itemize}

\noindent\textbf{\textsc{Semana 14}}\\[-0.5cm]
\begin{itemize}
\item Temas: Soluci\'on de sistemas de ecuaciones lineales. Ajustes
  por m\'inimos cuadrados. \\[-0.6cm] 
%\item Lecturas preparatorias:  Cap\'itulo 8 (Matrix Equation Solutions)
%  del libro de Landau. \\[-0.6cm]
\end{itemize}


\noindent\textbf{\textsc{Semana 15}}\\[-0.5cm]
\begin{itemize}
\item Temas: Autovalores y autovectores. 
An\'alisis de Componentes Principales. \\[-0.6cm]
%\item Lecturas preparatorias: Secciones 6.3.1 (Principal Component
%  Regression) y 10.2 (Principal Component Analysis) del libro ISL.\\[-0.6cm]
\end{itemize}

\noindent\textbf{\textsc{Semana 16}}\\[-0.5cm]
\begin{itemize}
\item Temas. Campos Vectoriales. Derivadas parciales. Gradientes. Divergencia.\\[-0.6cm]  
\end{itemize}


%\vspace*{0.5cm} 
%\stepcounter{mysection}
%\noindent\textbf{\large \Roman{mysection} \quad
  Metodolog\'ia}\\[-0.2cm] 

%El curso es de 3 cr\'editos que corresponden a 9 horas de dedicaci\'on.
%Habr\'a $5$ horas presenciales distribu\'idas en $2\times 1.5$ horas de magistral 
%y $2$ horas de complementaria.
%Las $4$ horas restantes corresponden a trabajo individual para la preparaci\'on de los contenidos de la semana. La propuesta busca implementar una estrategia de orientaci\'on guiada para el aprendizaje de las Ciencias Computacionales.

%\begin{itemize}
%    \item Antes de cada clase magistral al estudiante se le propone una lectura, un video
%    explicativo de $15$ minutos y ejercicios para resolver. 
%    El tiempo de estudio y no debe exceder el tiempo previsto para trabajo individual.
%    \item Una secci\'on magistral, repartida en 2 sesiones semanales de 1.5 horas cada una.
%    En cada clase se presentan los conceptos te\'oricos centrales para luego mostrar su
%    implementaci\'on computacional.
 %   \item Secciones complementarias (m\'aximo 15 estudiantes por profesor), correspondiente a
 %   una \'unica sesi\'on de $2$ horas. El enfoque es pr\'actico, el \'enfasis esta en el
 %   desarrollo de ejercicios por parte de los estudiantes. Los ejercicios de esta secci\'on
%    deben ser dise\~nados por el profesor de la magistral para resolver un problema en cada
%    secci\'on.
%    El ejercicio debe tomar $1$ hora de an\'alisis y pr\'actica con desarrollo guiado por el
%    monitor y $1$ hora de desarrollo individual de parte del estudiante para entrega 
%    al final de la clase.
%\end{itemize}
%Se describen las tecnicas y metodos para el desarrollo exitoso del curso.

%\noindent\normalsize 


%\vspace*{0.5cm} 
%\stepcounter{mysection}
%\noindent\textbf{\large \Roman{mysection} \quad Criterios de
%  evaluaci\'on}\\[-0.2cm] 

%\begin{itemize}
%    \item Ejercicios de programaci\'on y quices resueltos en la Magistral: $50\%$.  
%    \item Ejercicios de programaci\'on y quices resueltos en el Laboratorio Complementario: $30\%$. 
%    \item Examen Final (escrito y de programaci\'on): $20\%$.
%\end{itemize}

%Todos los ex\'amenes, talleres y ejercicios son
%\textbf{individuales}.  
%Si en las entregas se detecta que el trabajo no fue
%individual (esto incluye colaboraci\'on con personas no inscritas en
%el curso, i.e. a trav\'es de ``monitor\'ias'') se llevar\'a el caso a
%comit\'e disciplinario y la nota del curso queda como Pendiente
%Disciplinario hasta que el comit\'e tome alguna decisi\'on. 

%Todas las entregas de talleres y ejercicios se har\'an a trav\'es de
%SICUA.  {\bf No se aceptar\'a ninguna tarea por fuera de esa
%  plataforma}, a menos que ocurra un una falla en los servidores de
%SICUA que afecte a {\bf todos} los estudiantes del curso.


%\vspace*{0.5cm} 
%\stepcounter{mysection}
%\noindent\textbf{\large \Roman{mysection} \quad Perfil de docente de M\'etodos computacionales}\\[-0.2cm] 

%{\color{red} \textbf{FALTA DETERMINAR EL PERFIL DOCENTE}}

\vspace*{0.5cm} 

\stepcounter{mysection}
\noindent\textbf{\large \Roman{mysection} \quad
  Bibliograf\'ia}\\[-0.2cm] 

%Indicar los libros y la documentacion guia.


\noindent\normalsize Bibliograf\'ia principal:

\begin{itemize}
\item
\textit{Software Carpentry: Python Testing}
\url{http://katyhuff.github.io/python-testing/}

\item
\textit{A survey of Computational Physics - Enlarged Python Book}
. R. H. Landau, M. J. P\'aez, C. C. Bordeianu. WILEY. 2012.
\url{https://psrc.aapt.org/items/detail.cfm?ID=11578}

\item
\textit{Data Analysis: A Bayesian Tutorial.} D. S. Sivia,
J. Skilling. Second Edition, Oxford Science Publications. 2006.

\item 
\textit{Statistical Mechanics: Algorithms and Computations.}
W. Krauth, Oxford Univ. Press. 

\item
\textit{An Introduction to Statistical Learning.} G. James, D. Witten,
T. Hastie, R. Tibshirani,
Springer. \url{http://www-bcf.usc.edu/~gareth/ISL/} 


\item Software Carpentry: \url{http://software-carpentry.org/}
\item C++ Tutorial: \url{https://www.tutorialspoint.com/cplusplus/}
\end{itemize}

\noindent\normalsize Bibliograf\'ia secundaria:
\begin{itemize}
\item
\textit{Elements of Scientific Computing}
Tveito A., Langtangen H.P., Nielsen B.F., Cai X. Spinger. 2010.


\item 
\textit{Introduction to Computation and Programming Using Python},
Guttag, J. V. The MIT Press. 2013.

\end{itemize}


\end{document}
